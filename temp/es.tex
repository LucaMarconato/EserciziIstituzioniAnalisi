\documentclass[12pt,a4paper]{article} %potrebbe servire fleqn per l'allineamento a sinistra con multline
\usepackage{amsmath}
\usepackage{amsthm}
\usepackage{amsfonts}
\usepackage{amssymb}
\newif\ifdraft
%\draftfalse % <-------- draftfalse = VERSIOyE DRAFT
\drafttrue % <-------- drafttrue = VERSIOyE FIyALE
\ifdraft
\usepackage{graphicx}
\else
\usepackage[draft]{graphicx}
\fi

\usepackage{mathrsfs}
\usepackage{hyperref}
\usepackage{pifont}
\usepackage{mathtools}
\usepackage[utf8]{inputenc}
%\usepackage[top=1.0in, bottom=1.0in, left=0.5in, right=0.5in]{geometry}
\usepackage{xcolor}
\usepackage{titlesec}
\usepackage[italian]{babel}
\usepackage[pages=some]{background} %per il polloPallido in background
\ifdraft
\else
\usepackage[right]{showlabels}
\fi
%\usepackage[lite]{mtpro2} %for widetilde
%\usepackage[left=1.5cm,right=1.5cm,top=1.5cm,bottom=1.5cm,a4paper]{geometry} %per \newgeometry
%\usepackage[left=2.5cm,right=2.5cm,top=1.5cm,bottom=2cm,a4paper]{geometry} %per \newgeometry
\usepackage{geometry}
\usepackage{tikz}
\usepackage{pgfplots} %per fare plot con tikz
\usetikzlibrary{matrix,arrows,decorations.pathmorphing,calc}
\usepackage{makeidx}                       % Indice analitico

%\usepackage{natbib} %per \bibliographystyle{apa}
\makeindex

%frontespizio
\usepackage[]{microtype}
\usepackage{titling}
\DeclareRobustCommand{\univsc}[1]{\Large\textsc{\textls[75]{#1}}}
\DeclareRobustCommand{\titlesc}[1]{\linespread{1.2}\LARGE{\textls[35]{\uppercase{#1}}\linespread{1}}}
\DeclareRobustCommand{\namesc}[1]{\large\textsc{\textls[35]{#1}}} 




%% Environment
\theoremstyle{plain}
\newtheorem{exercise}{Exercise}







%% Shortcuts
% Generic
\newcommand\de{\ensuremath{\,\mathrm d}}                              %For integrals

% Gothics
\newcommand\A{\ensuremath{\mathfrak A}}                               %Ring
\newcommand\E{\ensuremath{\mathfrak E}}                               %
\newcommand\M{\ensuremath{\mathfrak M}}                               %
\renewcommand\P{\ensuremath{\mathfrak P}}                             %Semiring or pseudoring

% Standard sets



\newcommand\F{\ensuremath{\mathbb F}}                                 %Generic field F
\newcommand\K{\ensuremath{\mathbb K}}                                 %Generic field K
\newcommand\N{\ensuremath{\mathbb N}}                                 %Natural numbers
\newcommand\Q{\ensuremath{\mathbb Q}}                                 %Rationals
\newcommand\R{\ensuremath{\mathbb R}}                                 %Real numbers
\newcommand\Z{\ensuremath{\mathbb Z}}                                 %Integers


%% Notation uniformity
% Generic
\newcommand\st{\colon}                                                %"such that"
\newcommand\powerset{\ensuremath{\mathbb P}}                          %Power set
\newcommand\compl[1]{\~#1}                                            %Complement

% Intervals of real numbers: c for closed, o for open
\newcommand\icc[2]{\ensuremath{\left[\,#1,\,#2\,\right]}}
\newcommand\ico[2]{\ensuremath{\left[\,#1,\,#2\,\right[}}
\newcommand\ioc[2]{\ensuremath{\left]\,#1,\,#2\,\right]}}
\newcommand\ioo[2]{\ensuremath{\left]\,#1,\,#2\,\right[}}

\title{Lavoro domestico}
\author{Luca Marconato}

\begin{document}
\maketitle
\begin{exercise}
  Let us show that
  \begin{subequations}
    \begin{align}    
      \mathcal{B} (\mathbb{R}) & = \sigma \left(\left\{ \ioo{a}{b} \st a, b \in \mathbb{R} \right\} \right) = \sigma \left(\left\{\icc{a}{b} \st a, b \in \mathbb{R} \right\}\right) \\
      & = \sigma \left(\left\{\ioc{a}{b} \st a, b \in \Q \right\}\right) = \sigma (\E_i)
    \end{align}
  \end{subequations}
  where
  \begin{enumerate}
    \item $\E_{1}\coloneqq \{\ioo{a}{+\infty} \st a \in \mathbb{R}\}$;
    \item $\E_{2}\coloneqq \{\ico{a}{+\infty} \st a \in \mathbb{R}\}$;
    \item $\E_{3}\coloneqq \{\ioo{+\infty}{a} \st a \in \mathbb{R}\}$;
    \item $\E_{4}\coloneqq \{\ioc{+\infty}{a} \st a \in \mathbb{R}\}$;
  \end{enumerate}
\end{exercise}
\begin{proof}
  Let us call $\sigma({\A_1}), \sigma({\A_2}), \sigma({\A_3})$ the three $\sigma$-ring that appear in the statement of this exercise that still do not have a name.
  First, let us observe that the various $\sigma$ rings we deal with are not only $\sigma$-rings but also $\sigma$-algebras.
  In fact for each of them $\R$ can be written as a countable union of its elements.
  Since the set of all open intervals is a subset of the family of the open sets, considering the generated $\sigma$-rings, we obtain that $\A_1 \subseteq \mathcal{B}(\R)$. 
  The other inclusion follows from the fact that an open set is by definition a numerable union of open intervals.
  $\A_1 = \A_2$ because every open interval can be written as the complementary of a closed set and vice versa.
  Let us now prove that $\A_2 = \A_3$.
  The key point is to notice that every $[a,b]$ can be written as the countable intersection $\bigcap \left\lbrace \ioc{a-\frac{1}{n}}{b}\right\rbrace$, that every $\ioc{a}{b}$ can be written as the countable union $\bigcup \left\lbrace \icc{a+\frac{1}{n}}{b}\right\rbrace$, and to remember that a $\sigma$-algebra contains all the countable unions and countable intersections.

  Clearly $\E_1 = \E_4$ and $\E_2 = \E_3$, because every element of each first set can it is the complementary of an element of each second set, and vice versa.
  The inclusion $\E_1 \subseteq \A_1$ follows by the fact that $\ioo{a}{+\infty}$ can be written as $\ioo{a}{a + 2} \cup \bigcup \ioo{a+n}{a+n+2}$. 
  Also the other inclusion is true because $\ioo{a}{b} = \ioo{a}{+\infty} \cap \bigcup \widetilde{\ioo{b-\frac{1}{n}}{+\infty}}$.

It remains to prove that $\E_1 = \E_2$. 
This can be done observing that $\ioo{a}{+\infty} = \bigcup \ico{a+\frac{1}{n}}{+\infty}$ and that $\ico{a}{+\infty} = \bigcap \ioo{a-\frac{1}{n}}{+\infty}$.
\end{proof}

\end{document}
%%% Local Variables:
%%% mode: latex
%%% TeX-master: t
%%% End:
