\documentclass[../main.tex]{subfiles} 
\begin{document}

\exercise[10/10/2016]{1} %g2
Dimostrare che
\begin{equation}
  \begin{aligned}    
  \label{eq:1}
  \Bor{\R} &=   \sigma(\left{]a,b[\colon a, b \in \R\right}) \\
  & = \sigma(\left{]a,b[\colon a, b \in \Q\right}) \\
  &= \sigma(\left{]a,b]\colon a, b \in \Q\right}) \\
  &= \sigma(\E_{i})
  \end{aligned}
\end{equation}
dove
\begin{itemize}
\item $\E_{1}\coloneqq \{]a,+\infty[:a \in \R\}$;
\item $\E_{2}\coloneqq \{[a,+\infty[:a \in \R\}$;
\item $\E_{3}\coloneqq \{]+\infty,b[:b \in \R\}$;
\item $\E_{4}\coloneqq \{]+\infty,b]:b \in \R\}$.
\end{itemize}

\solution

\exercise
Sia $  \emptyset \neq \E_{0} \subseteq \P(X)$ e per $\alpha$ ordinale sia 
\begin{equation}
  \label{eq:2}
  \E_{\alpha}\coloneqq \left{ \bigcup_{n\in \N} (A_{n}\setminus B_{n} \: a_{n},B_{n} \in \bigcup_{\beta<\alpha}\E_{\beta})\right}.
\end{equation}
Dimostrare allora che $\sigma(\E)_{0}=\bigcup_{\alpha<\omega_{1}}\E_{\alpha}$.
\end{document}

%%% Local Variables:
%%% mode: latex
%%% TeX-master: "../main"
%%% End:
