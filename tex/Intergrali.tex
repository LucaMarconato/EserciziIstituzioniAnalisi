\documentclass[../EserciziIstituzioniAnalisi.tex]{subfiles} 
\begin{document}
\begin{exercise}[28/10/2017]
  Sia $f$ $\mu$-misurabile e positiva, sia
  \begin{equation*}
    \forall A \in \gA\ \nu(A)\coloneqq \int f \chi_A \de \mu
  \end{equation*}
  Dimostrare che $\nu\colon\gA\rightarrow \icc{0}{+\infty}$ è una misura $\sigma$-additiva.
\end{exercise}
\begin{proof}
  Dimostriamo intanto l'additività: sia $C=A\sqcup B$; allora $f\chi_C=f\chi_A+f\chi_B$ e quindi $\nu(C)=\int f\chi_C \de \mu=\int f\chi_A \de \mu +\int f\chi_B \de \mu=\nu(A)+\nu(B)$ per la linearità dell'integrale se tutti i termini sono finiti, oppure perché per $0\leq c\leq +\infty,\ c+\infty=+\infty$.
  
  Sia ora $(A_n)_{n \in \N}\subseteq \gA$ una successione di insiemi disgiunti, e $A=\bigsqcup_{n\in\N}A_n$. Sia $B_n=\bigcup_{j\leq n}A_j$. Per la finita additività sappiamo che
  \begin{equation*}
    \int f\chi_{B_n} \de \mu=\sum_{j\leq n} \int f \chi_{A_j} \de \mu
  \end{equation*}
  inoltre $\lim_{n\to +\infty} f\chi_{B_n}=f\chi_A$\footnote{questo limite e i prossimi esistono perché consideriamo successioni crescenti}, quindi dal lemma di Fatou
  \begin{equation*}
    \int f\chi_A \de \mu\leq \lim_{n \to +\infty} \int f\chi{B_n} \de \mu=\lim_{n\to +\infty} \sum_{j\leq n} \int f \chi_{A_j} \de \mu=\sum_{n\in \N} \int f \chi_{A_n} \de \mu
  \end{equation*}
  Infine, dato che $B_n\uparrow A,\ f\chi_{B_n}\leq f\chi_A$, e quindi per la monotonia dell'integrale $\int f \chi_A \de \mu \geq \int f\chi_{B_n}\de \mu$ e
  \begin{equation*}
    \int f \chi_A \de \mu \geq \sup_{n\in\N} \int f\chi_{B_n} \de \mu = \lim_{n\to +\infty} \int f\chi_{B_n} \de \mu
  \end{equation*}
  Questa disuguaglianza, combinata con quella che deriva dal lemma di Fatou, ci da la tesi.  
\end{proof}
\begin{exercise}[3/11/2016]
  Trovare dei controesempi alle seguenti:
  \begin{enumerate}
    \item Il lemma di Fatou enunciato con $\limsup$ al posto del $\liminf$
    \item Il lemma di Fatou enunciato con l'uguaglianza.
  \end{enumerate}
\end{exercise}
\begin{proof}
  Usiamo in entrambi i casi la misura di Lebesue su $\R$.
  \begin{itemize}
    \item Sia $f_{2n}=\charf{\ico{0}{1/2}},f_{2n+1}=\charf{\icc{1/2}{1}}$. Allora $\int f_{2n} \de \mu=\int f_{2n+1} \de \mu=1/2$, quindi $\limsup \int f_n \de \mu=1/2$, mentre $\limsup f_n=\charf{\icc{0}{1}}$ e quindi $\int\limsup f_n \de \mu=1$.
    \item Sia $f_n=n \charf{\icc{0}{1/n}}$. Allora $\int f_n \de \mu=1$, e quindi $\liminf \int f_n \de \mu=1$; inoltre $\liminf f_n=0$ q.o. e quindi $\int\liminf f_n\de \mu=0$.
  \end{itemize}
\end{proof}
\end{document}

%%% Local Variables:
%%% mode: latex
%%% TeX-master: "../EserciziIstituzioniAnalisi"
%%% End:
