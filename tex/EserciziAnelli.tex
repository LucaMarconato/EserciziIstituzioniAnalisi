\documentclass[../EserciziIstituzioniAnalisi.tex]{subfiles} 
\begin{document}

\begin{exercise}[2016-10-10-1]
  Dimostrare che
  \begin{subequations}
    \begin{align}    
      \Bor (\R) & = \sigma \left(\left\{ \ioo{a}{b} \st a, b \in \R \right\} \right) \\
      & = \sigma \left(\left\{\ioo{a}{b} \st a, b \in \R \right\}\right) \\
      & = \sigma \left(\left\{\ioc{a}{b} \st a, b \in \Q \right\}\right) \\
      & = \sigma (\E_i)
    \end{align}
  \end{subequations}
  dove
  \begin{enumerate}
    \item $\E_{1}\coloneqq \{\ioo{a}{+\infty} \st a \in \R\}$;
    \item $\E_{2}\coloneqq \{\ico{a}{+\infty} \st a \in \R\}$;
    \item $\E_{3}\coloneqq \{\ioo{+\infty}{a} \st a \in \R\}$;
    \item $\E_{4}\coloneqq \{\ioc{+\infty}{a} \st a \in \R\}$;
  \end{enumerate}
\end{exercise}
\begin{proof}
  
\end{proof}

\begin{exercise}[insert date]  
  Sia $\emptyset \neq \E_0 \subseteq \powerset(X)$ e per $\alpha \in \Ord$ sia 
\begin{equation}
  \E_{\alpha}\coloneqq \left\{ \bigcup_{n\in \N} \left(A_n\setminus B_n \st A_n,B_n \in \bigcup_{\beta<\alpha}\E_{\beta}\right)\right\}.
\end{equation}
Dimostrare allora che $\sigma(\E_0)=\bigcup_{\alpha<\omega_1}\E_{\alpha}$.
\end{exercise}
\begin{proof}
  Dimostriamo le due inclusioni.
  \begin{itemize}
    \item[$\supseteq$] Si può dimostrare utilizzando il principio di induzione transfinita, nella forma
    $$P(0)\land \forall \alpha>0 \left(\forall \beta<\alpha P(\beta) \implies P(\alpha) \right) \implies \forall \alpha P(\alpha)$$
    sulla proposizione $\sigma(\E_0)\supseteq\E_{\alpha}$.
    \begin{itemize}
      \item[$\alpha=0$] Vero per definizione di ($\sigma$-)anello generato.
      \item[$\alpha>0$] Per ipotesi induttiva $\sigma(\E_0)\supseteq\E_{\beta}\, \forall \beta<\alpha$, quindi $\sigma(\E_0)\supseteq\bigcup_{\beta<\alpha}\E_{\beta}$. Ma $\sigma(\E_0)$ è chiuso per differenza e unione numerabile, quindi date due successioni $A_n, B_n \in \bigcup_{\beta<\alpha}\E_{\beta}$, abbiamo che $\forall n \in \N\, C_n=A_n\setminus B_n\in \sigma(\E_0)$, e $\bigcup_{n\in \N}C_n \in \sigma(\E_0)$. Per ipotesi un generico elemento di $\E_{\alpha}$ si scrive in questa forma, e dunque $\sigma(\E_0)\supseteq\E_{\alpha}$.
    \end{itemize}
    Allora $\sigma(\E_0)$ contiene ogni $\E_{\alpha}$, e quindi anche la loro unione.
    \item[$\subseteq$] Basta dimostrare che $S\coloneqq\bigcup_{\alpha<\omega_1}\E_{\alpha}$ è un anello.
    \begin{itemize}
      \item[Differenza] Siano $A,B\in S$; allora $\exists \alpha_1, \alpha_2<\omega_1 \st A\in \E_{\alpha_1}, B\in \E_{\alpha_2}$, e sia $\alpha=\min\{\alpha_1, \alpha_2\}$. allora per ipotesi $A\B\in\E_{\alpha+1}$, con $\alpha+1<\omega_1$, perché $\omega_1$ non è un successore.      
      \item[Unione] Sia $(A_n)_{n\in\N}$ una successione si elementi di $S$ 
    \end{itemize}
  \end{itemize}
\end{proof}
\end{document}

%%% Local Variables:
%%% mode: latex
%%% TeX-master: "../EserciziIstituzioniAnalisi"
%%% End:
