\documentclass[../EserciziIstituzioniAnalisi.tex]{subfiles} 
\begin{document}

\begin{exercise}[2016-10-20]
  Sia $\gA$ un anello, $\eta\colon\gA\to\Rpinf$ una submisura. Sia come al solito $\gN(\eta)=\left\{A\in\gA\st\eta(A)=0\right\}$. Dimostrare che
  \begin{enumerate}
    \item Se $\gN(\eta)$ è un ideale di $\gA$ e $\eta$ è una $\sigma$-submisura, allora $\gN$ è un $\sigma$-ideale.
    \item Se $A,B\in \gA$ e $A\cap B\in \gN(\eta)$, allora $\eta(A)=\eta(A\cap N)=\eta(B)$.
    \item Se $A\in \gA,\, N\in \gN(\eta)$, allora $\eta(A)=\eta(A\cup N)=\eta(A\Delta N)=\eta(A\setminus N)$
  \end{enumerate}
\end{exercise}
\begin{proof}
  
\end{proof}
\begin{exercise}[2016-10-20]
  Sia $\gA\subseteq\PS(X)$ un $\sigma$-anello, $\mu\colon \gA\to\Rpinf$ una misura $\sigma$-additiva.
  Sia $\gN\coloneqq\left\{N\st\exists A\in \gA, N\subseteq A, A\in \gN(\mu)\right\}$.

  \begin{enumerate}
    \item Dimostrare che $\gN$ è un $\sigma$-ideale in $\PS(X)$.
    \item Sia $\gL\coloneqq\ring(\gA\cup\gN)$. Dimostrare che $\gL=\left\{A\Delta N\st A\in\gA,N\in\gN\right\}=\left\{A\sqcup N\st A\in\gA,N\in\gN\right\}$ 
  \end{enumerate}
\end{exercise}
\begin{proof}
  \begin{enumerate}
    \item Sia $\hat N\subseteq N \in \gN$; per ipotesi $\exists A\in \gA \st N \subseteq A,\, \mu(A)=0$, ma allora anche $\hat N \subseteq A$, e quindi $\hat N \in \gN$.

    Sia ora $(N_n)_{n \in\N}$ una successione di elementi di $\gN$; per ipotesi $\forall n \exists A_n\in \gA \st \mu(A_n)=0,\, N_n\in A_n$; allora $\bigcup_{n \in \N} N_n\subseteq \bigcup_{n \in \N} A_n$ e per la $\sigma$-additività di $\mu$, $\mu(\bigcup_{n \in \N} A_n)\leq \sum_{n\in\N}\mu(A_n)=0$ e quindi per definizione $\bigcup_{n \in \N} N_n\in \gN$.
    \item Poniamo
    \begin{equation*}
      \begin{aligned}
        \mathfrak{U}\coloneqq \left\{A\Delta N\st A\in\gA,N\in\gN\right\}\\
        \mathfrak{T}\coloneqq \left\{A\sqcup N\st A\in\gA,N\in\gN\right\}
      \end{aligned}
    \end{equation*}
    Mostriamo che $\mathfrak{U}=\mathfrak{T}$
    Siano $A\in\gA,N\in\gN$.
    \begin{itemize}
      \item[$\subseteq$] Consideriamo $A\Delta N=(A\setminus N)\sqcup (N\setminus A)$. Per ipotesi $\exists C\in\gA$ di misura nulla tale che $N\subseteq C$. Sia $B=A\cap C$; allora
      \begin{equation*}
        B\in\gN, B\in\gA, \mu(B)=0;
      \end{equation*}
      inoltre
      \begin{equation*}
        A\setminus N=(A\setminus B)\sqcup (B\setminus N).
      \end{equation*}
      Allora abbiamo
      \begin{equation*}
        A\Delta N=(A\setminus B)\sqcup (B\setminus N)\sqcup (N\setminus A)
      \end{equation*}
      dove $A\setminus B\in \gA$ e $(B\setminus N)\sqcup (N\setminus A)\in \gN$, e dunque $A\Delta N\in \mathfrak{T}$.
      \item[$\supseteq$] Sia ora $A\cap N=\emptyset$. Allora $A\sqcup N=A\Delta N$ e la tesi è provata.
    \end{itemize}
    Mostriamo ora $\mathfrak{U}=\gL$.
    \begin{itemize}
      \item[$\supseteq$] È vero perché per definizione ogni anello è chiuso per differenza simmetrica.
      \item[$\subseteq$] Basta mostrare che $\mathfrak{U}$ è un anello e che contiene $\gA\cup\gN$; $\gA\subseteq\mathfrak{U}$ perché ogni elemento $A$ di $\gA$ si scrive come $A\Delta \emptyset$, e analogamente  $\gN\subseteq\mathfrak{U}$ Perché ogni elemento $N$ di $\gN$ si scrive come $N\Delta \emptyset$.
      
      Per provare che $\mathfrak{U}$ è un anello mostriamo che $\mathfrak{T}$ è chiuso per unione disgiunta e differenza.
      \begin{itemize}
        \item[$\sqcup$] Siano $A\sqcup N, A_1\sqcup N_1\in\mathfrak{U}$ disgiunti; allora $(A\sqcup N) \sqcup (A_1\sqcup N_1)= (A \sqcup A_1) \sqcup (N\sqcup N_1)\in\mathfrak{U}$, perché l'unione è associativa e commutativa.
        \item[$\setminus$] Mostriamo la differenza di due elementi di $\mathfrak{T}=\mathfrak{U}$ appartiene a $\mathfrak{U}$. Siano $A\sqcup N, A_1\sqcup N_1\in\mathfrak{U}$; allora $(A\sqcup N) \setminus (A_1\sqcup N_1)=(A\setminus A_1\setminus N_1) \sqcup (N\setminus A_1\setminus N_1)$.

        Ma il secondo insieme dell'unione è un sottoinsieme di $N$, e quindi appartiene all'ideale $\gN\subseteq \mathfrak{U}$.

        Inoltre sia $N_1\subseteq B\in \gA, \mu(B)=0$ contenuto in $A\setminus A_1$. $A\setminus A_1\setminus N_1=A\setminus A_1\setminus B \sqcup B\setminus N_1$ dove e quindi il primo insieme dell'unione appartiene ad $\mathfrak{U}$.

        Dato che abbiamo già dimostrato che $\mathfrak{U}$ è chiuso per unioni disgiunte, è chiuso anche per differenza.
      \end{itemize}
    \end{itemize}
  \end{enumerate}
\end{proof}
\begin{exercise}[27/10/2016]
  Definiamo $\overline \mu(A \Delta N) \coloneqq \mu (A) \,\forall A \in \gA,N\in\N$.
  Dimostrare che $\overline \mu$ è ben definita, $\sigma$-additiva, completa e che $\overline \mu\mid_{\gA}=\mu$.
\end{exercise}
\begin{remark}
  La misura $\overline\mu$ così definita si dice completamento di $\mu$.
\end{remark}
\begin{proof}
  Definiamo $\mathfrak{U}\coloneqq \left\{A\Delta N\st A\in\gA,N\in\gN\right\}$. Gli ultimi due punti sono ovvi prendendo $A=\emptyset$ e $N=\emptyset$ rispettivamente. Dimostriamo quindi i primi due.
  \begin{itemize}
    \item Dimostriamo che $\overline\mu$ è ben definita. Sia $A_1 \Delta N_1=A_2\Delta N_2$, con $A_1,A_2\in \gA; N_1,N_2\in\gN$. Allora $\exists B_1, B_2\in \gA\st \mu(B_1)=\mu(B_2)=0, N_1\subseteq B_1, N_2\subseteq B_2$. Quindi $A_1\subseteq A_2\cup B_1 \cup B_2$ e per l'additività della misura $\mu(A_1)\leq \mu(A_2)+\mu(B_1)+\mu(B_2)=\mu(A_2)$; allo stesso modo anche $\mu(A_2)\leq \mu(A_1)$ e quindi $\mu(A_1)=\mu(A_2)$.
    \item Per dimostrare la $\sigma$-additività utilizziamo l'esercizio precedente: dati $A,N$ come nelle ipotesi, $\exists \overline A\in \gA; \overline N\in\gN \st A \Delta N= \overline A \sqcup \overline N$. Inoltre, per la buona definizione di $\overline \mu$, e poiché $\overline A \sqcup \overline N=\overline A \Delta \overline N$, abbiamo che $\mu(A)=\mu(\overline A)$. Sia quindi $\left(A_k \Delta N_k \right)_{k\in \N}$ una successione di elementi disgiunti in $\mathfrak{U}$. Dato che $\gN$ è un $\sigma$-anello $\exists \hat A\st \mu(\hat A)=0, \bigcup_{k \in \N} N_k\subseteq \hat A$. Inoltre detti $\hat A_k\coloneqq A_k \setminus \hat A$, si ha $\mu(A_k)=\mu(\hat A_k)$.
    \begin{equation*}
      \begin{aligned}
        \overline \mu\left(\bigsqcup_{k\in \N} \left( A_k \Delta N_k\right)\right)&=\overline \mu\left(\bigsqcup_{k\in \N} \left(\overline A_k \sqcup \overline N_k\right)\right)=\\
        &=\overline \mu\left(\bigsqcup_{k\in \N} \left(\hat A_k \sqcup \left(\overline N_k \cup\hat A \right)\right)\right)=\\
        &=\overline \mu\left(\hat A \sqcup \bigsqcup_{k\in \N} \hat A_k \right)=\\
        &=\mu\left(\bigsqcup_{k\in \N} \hat A_k\right)=\sum_{k \in \N}\mu\left(\hat A_k\right)=\sum_{k \in \N}\mu\left(A_k\right)=\\
        &=\sum_{k \in \N}\overline\mu\left(A_k \Delta N_k\right)
      \end{aligned}
    \end{equation*}
    che è la $\sigma$-additività.
  \end{itemize}
\end{proof}

\begin{exercise}[14/10/2016]
  Sia $\gA$ un'algebra, e $S(\gA)$ l'insieme delle funzione semplici su $\gA$.
  
  Sia $E=\left(S(\gA,\norm{\cdot}_s)\right)$, $T\in E'=\{\xi \in E* continue\}$.
  Sappiamo che $\exists \mu\st T(f)=\int f \de \mu$.
  
  Calcolare la norma $\norm{T}$ operatoriale in funzione della misura.
\end{exercise}
\begin{proof}
  
\end{proof}
\begin{exercise}[14/10/2016]
  Sia $\gA$ un anello, $\mu\colon\gA\to X$
  Dire quali delle seguenti sono equivalenti, nei due casi $X=\R$ e $X=\icc{0}{+\infty}$:
  \begin{enumerate}
    \item $\mu$ è $\sigma$-additiva
    \item $\forall (A_n)_{n\in\N}\subseteq \gA,\ A_n \uparrow A\in \gA$ si ha $\mu(A_n)\rightarrow \mu(A)$
    \item $\forall (A_n)_{n\in\N}\subseteq \gA,\ A_n \downarrow A\in \gA$ si ha $\mu(A_n)\rightarrow \mu(A)$ 
    \item $\forall (A_n)_{n\in\N}\subseteq \gA,\ A_n \downarrow \emptyset$ si ha $\mu(A_n)\rightarrow 0$
  \end{enumerate}
\end{exercise}
\begin{proof}
  
\end{proof}
\end{document}

%%% Local Variables:
%%% mode: latex
%%% TeX-master: "../EserciziIstituzioniAnalisi"
%%% End:
