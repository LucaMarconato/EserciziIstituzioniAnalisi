\documentclass[../EserciziIstituzioniAnalisi.tex]{subfiles} 
\begin{document}

\begin{exercise}[2016-10-20-1]
  Sia $\gA$ un anello, $\eta\colon\gA\to\Rpinf$ una submisura. Sia come al solito $\gN(\eta)=\left\{A\in\gA\st\eta(A)=0\right\}$. Dimostrare che
  \begin{enumerate}
    \item Se $\gN(\eta)$ è un ideale di $\gA$ e $\eta$ è una $\sigma$-submisura, allora $\gN$ è un $\sigma$-ideale.
    \item Se $A,B\in \gA$ e $A\cap B\in \gN(\eta)$, allora $\eta(A)=\eta(A\cap N)=\eta(B)$.
    \item Se $A\in \gA,\, N\in \gN(\eta)$, allora $\eta(A)=\eta(A\cup N)=\eta(A\Delta N)=\eta(A\setminus N)$
  \end{enumerate}
\end{exercise}
\begin{proof}
  
\end{proof}
\begin{exercise}[2016-10-20-2]
  Sia $\gA\subseteq\PS(X)$ un $\sigma$-anello, $\mu\colon \gA\to\Rpinf$ una misura $\sigma$-additiva.
  Sia $\gN\coloneqq\left\{N\st\exists A\in \gA, N\subseteq A, A\in \gN(\mu)\right\}$.
  
  Dimostrare che $\gN$ è un ideale in $\PS(X)$.
\end{exercise}
\begin{proof}
  
\end{proof}
\end{document}

%%% Local Variables:
%%% mode: latex
%%% TeX-master: "../EserciziIstituzioniAnalisi"
%%% End:
