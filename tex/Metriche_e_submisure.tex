\documentclass[../EserciziIstituzioniAnalisi.tex]{subfiles} 
\begin{document}

\begin{exercise}[2016-10-20-1]
  Sia $\gA$ un anello, $\eta\colon\gA\to\Rpinf$ una submisura. Sia come al solito $\gN(\eta)=\left\{A\in\gA\st\eta(A)=0\right\}$. Dimostrare che
  \begin{enumerate}
    \item Se $\gN(\eta)$ è un ideale di $\gA$ e $\eta$ è una $\sigma$-submisura, allora $\gN$ è un $\sigma$-ideale.
    \item Se $A,B\in \gA$ e $A\cap B\in \gN(\eta)$, allora $\eta(A)=\eta(A\cap N)=\eta(B)$.
    \item Se $A\in \gA,\, N\in \gN(\eta)$, allora $\eta(A)=\eta(A\cup N)=\eta(A\Delta N)=\eta(A\setminus N)$
  \end{enumerate}
\end{exercise}
\begin{proof}
  
\end{proof}
\begin{exercise}[2016-10-20-2]
  Sia $\gA\subseteq\PS(X)$ un $\sigma$-anello, $\mu\colon \gA\to\Rpinf$ una misura $\sigma$-additiva.
  Sia $\gN\coloneqq\left\{N\st\exists A\in \gA, N\subseteq A, A\in \gN(\mu)\right\}$.

  \begin{enumerate}
    \item Dimostrare che $\gN$ è un ideale in $\PS(X)$.
    \item Sia $\gL\coloneqq\ring(\gA\cup\gN)$. Dimostare che $\gL=\left\{A\Delta N\st A\in\gA,N\in\gN\right\}=\left\{A\sqcup N\st A\in\gA,N\in\gN\right\}$ 
  \end{enumerate}
\end{exercise}
\begin{proof}
  \begin{enumerate}
    \item Sia $\hat N\subseteq N \in \gN$; per ipotesi $\exists A\in \gA \st N \subseteq A,\, \mu(A)=0$, ma allora anche $\hat N \subseteq A$, e quindi $\hat N \in \gN$.

    Consideriamo allora per $n \in \N\, N_n \in \gN$; per ipotesi $\forall n \exists A_n\in \gA \st \mu(A_n)=0,\, N_n\in A_n$; allora $\bigcup_{n \in \N} N_n\subseteq \bigcup_{n \in \N} A_n$ e per la $\sigma$-additività di $\mu$, $\mu(\bigcup_{n \in \N} A_n)\leq \sum_{n\in\N}\mu(A_n)=0$ e quindi per definizione $\bigcup_{n \in \N} N_n\in \gN$.
    \item Poniamo
    \begin{equation*}
      \begin{aligned}
        \mathfrak{U}\coloneqq \left\{A\Delta N\st A\in\gA,N\in\gN\right\}\\
        \mathfrak{T}\coloneqq \left\{A\sqcup N\st A\in\gA,N\in\gN\right\}
      \end{aligned}
    \end{equation*}
    Mostriamo che $\mathfrak{U}=\mathfrak{T}$
    Siano $A\in\gA,N\in\gN$.
    \begin{itemize}
      \item[$\subseteq$] Abbiamo che $A\Delta N=(A\setminus N)\sqcup (N\setminus A)$. Per ipotesi $\exists B\in\gA$ di misura nulla (e quindi $B\in\gN$) tale che $N\subseteq B$, e $A\setminus N=(A\setminus B)\sqcup (B\setminus N)$. Allora $A\Delta N=(A\setminus B)\sqcup (B\setminus N)\sqcup (N\setminus A)$, dove $A\setminus B\in \gA$ e $(B\setminus N)\sqcup (N\setminus A)\in \gN$, e dunque $A\Delta N\in \mathfrak{T}$.
      \item[$\supseteq$] Sia inoltre $A\cap N=\emptyset$. Allora $A\sqcup N=A\Delta N$ e la tesi è provata.
    \end{itemize}
    Mostriamo ora $\mathfrak{U}=\gL$.
    \begin{itemize}
      \item[$\supseteq$] È vero perché per definizione ogni anello è chiuso per differenza simmetrica.
      \item[$\subseteq$] Basta mostrare che $\mathfrak{U}$ è un anello e che contiene 
    \end{itemize}
  \end{enumerate}
\end{proof}
\end{document}

%%% Local Variables:
%%% mode: latex
%%% TeX-master: "../EserciziIstituzioniAnalisi"
%%% End:
