\documentclass[../EserciziIstituzioniAnalisi.tex]{subfiles} 
\begin{document}                %label:sgn
\begin{exercise}
  Sia $P,N$ una decomposizione di Hahn, $P_0\in\gA,\ N_0\coloneqq X\setminus P_0$.

  Allora $P_0,N_0$ è una decomposizione di Hahn se e solo se $P\Delta P_0\in\gN(\mu)$ se e solo se $N\Delta N_0\in\gN(\mu)$, dove $\gN(\mu)\coloneqq\{A\in\gA\st \forall B\in \gA, B\subseteq A \mu(B)=0\}$.
\end{exercise}
\begin{proof}
  
\end{proof}
\begin{exercise}
  \begin{gather*}
    \mu^+(A)=\sup\{\mu(M)\st M\in \gA, M\subseteq A\}\\
    \mu^-(A)=\inf\{\mu(M)\st M\in \gA, M\subseteq A\}\\
    \abs{\mu(A)}=\sup\left\{\sum_{i=1}^n\abs\mu(A_i)\st n\in \N, \gA\ni A_i \text{ disgiunti, }A=\bigcup_{i=1}^{n}A_i\right\}\\
    \gN(\mu)=\gN(\abs\mu)=\gN(\mu^+)\cap\gN(\mu^-)
  \end{gather*}
  (quindi $\gN(\mu)$ è un $\sigma$-ideale in  $\gA$).
  
  Se $\nu\colon \gA\to\Rpinf$ è una misura $\sigma$-additiva, $f\in\mathcal{L}_1(\nu)$, $\mu(A)\coloneqq\int_Af\de\nu$,
  allora $\mu^+(A)=\int_Af^+\de \nu,\ \mu^-(A)=\int_Af^-\de\nu,\ \abs(\mu)(A)=\int_A\abs{f}\de\nu$.
\end{exercise}
\begin{proof}
  
\end{proof}
\begin{exercise}[Controesempo a Radon-Nikodyn senza l'ipotesi di $\sigma$-finitezza]
  Sia $\alpha$ la misura di Lebesgue su $\Bor(\icc{0}{1})$, $\mu$ la counting measure su $\Bor(\icc{0}{1})$.
  
  Dimostrare che $\nexists\lambda,\nu\colon\Bor(\icc{0}{1})\to \Rpinf$ misure $\sigma$-additive con $\mu=\lambda+\nu, \lambda \ll \alpha, \nu \perp \alpha$.
\end{exercise}
\begin{proof}
  Siano per assurdo $\mu, \nu$ due misure che soddisfano le richieste.
  Dimostriamo allora che $\gN(\nu)=\emptyset$.

  Per $x\in\icc{0}{1}$ abbiamo $1=\mu(\{x\})=\lambda(\{x\})+\nu(\{x\})$, da cui $\nu(\{x\})=1$ dato che $\{x\}\in\gN(\alpha)\subseteq\gN(\lambda)$.
  Sia allora $\emptyset\neq B\in\Bor(\icc{0}{1})$; per monotonia della misura, $\nu(B)\geq\nu(\{x\})$ per $x\in B$, che esiste per ipotesi.

  Per definizione di singolarità tra misure $\exists N\in\gN(\nu),\, A\in\gN(\alpha)\st N\cup A=\icc{0}{1}$. Quindi deve essere $A=\icc{0}{1}$, che però non ha misura di Lebesgue nulla, e quindi non appartiene a $\gN(\alpha)$, assurdo.
\end{proof}
\end{document}
