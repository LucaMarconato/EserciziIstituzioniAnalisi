\documentclass[../EserciziIstituzioniAnalisi.tex]{subfiles} 
\begin{document}

We start with a lemma.
\begin{lemma}
  Every open set $A \subseteq \mathbb{R}$ can be written as a countable union of open intervals.
\end{lemma}
\begin{proof}
  First, we remind what a \textit{base for a topology} is.
  Let $(X,\tau)$ be a topological space. 
  A base for $\tau$ is a set $\mathcal{A} \subseteq \tau$ such that
  \begin{equation*}
    \forall T \in \tau \ \forall x \in T \ \exists A(x,T) \in \mathcal{A}\ x \in A \subseteq T.
  \end{equation*}
  If $T \in \tau$, using the notation of the previous definition, then we have that
  \begin{equation*}
    T = \bigcup_{x \in T} A(x,T).
  \end{equation*}
  In other words a base for $\tau$ is a subset $\mathcal{A}$ of $\tau$ such that every set of $\tau$ can be written as an union of elements in $\mathcal{A}$.

  Now, given any open set $T$ of $\mathbb{R}$ and given $x \in \mathbb{R}$ by definition exists an open interval contained in $T$ and containing $x$.
  It is easy to find an open interval with rational endpoints which contains $x$ and which is contained in $T$.
  This shows that the set
  \begin{equation*}
    \left\lbrace \ioo{a}{b}\colon a, b \in \mathbb{Q} \right\rbrace
  \end{equation*}
  is a base of the Euclidean topology.
  Since the base is countable, this proves that every open set of $\mathbb{R}$ can be written as a countable union of open intervals.
\end{proof}

\begin{exercise}[2016-10-10-1]
  Let us show that
  \begin{subequations}
    \begin{align}    
      \Bor{\mathbb{R}} & = \sigma \left(\left\{ \ioo{a}{b} \st a, b \in \mathbb{R} \right\} \right) = \sigma \left(\left\{\icc{a}{b} \st a, b \in \mathbb{R} \right\}\right) \\
      & = \sigma \left(\left\{\ioc{a}{b} \st a, b \in \Q \right\}\right) = \sigma(\gE_i)
    \end{align}
  \end{subequations}
  where
  \begin{enumerate}
    \item $\gE_{1}\coloneqq \{\ioo{a}{+\infty} \st a \in \mathbb{R}\}$;
    \item $\gE_{2}\coloneqq \{\ico{a}{+\infty} \st a \in \mathbb{R}\}$;
    \item $\gE_{3}\coloneqq \{\ioo{+\infty}{a} \st a \in \mathbb{R}\}$;
    \item $\gE_{4}\coloneqq \{\ioc{+\infty}{a} \st a \in \mathbb{R}\}$;
  \end{enumerate}
\end{exercise}
\begin{proof}
  Let us call $\sigma({\gA_1}), \sigma({\gA_2}), \sigma({\gA_3})$ the three $\sigma$-ring that appear in the statement of this exercise that still do not have a name.
  First, let us observe that the various $\sigma$ rings we deal with are not only $\sigma$-rings but also $\sigma$-algebras.
  In fact for each of them $\R$ can be written as a countable union of its elements.

  In order to prove the various equalities $\sigma(\mathfrak{M}) = \sigma(\mathfrak{N})$ we will first prove that $\mathfrak{M} \subseteq \mathfrak{N}$, thus obtaining $\sigma(\mathfrak{M}) \subseteq \sigma(\mathfrak{N})$ and the then we will prove that $\mathfrak{N} \subseteq \sigma(\mathfrak{M})$ concluding the proof.

  Since the set of all open intervals is a subset of the family of the open sets, considering the generated $\sigma$-rings, we obtain that $\sigma(\gA_1) \subseteq \Bor{R}$. 
  The other inclusion follows from the lemma we have seen before this exercise, which states that every open set can be written as a numerable union of open intervals.
  $\sigma(\gA_1) = \sigma(\gA_2)$ because every open interval can be written as the complementary of a closed set and vice versa.
  Let us now prove that $\sigma(\gA_2) = \sigma(\gA_3)$.
  The key point is to notice that every $[a,b]$ can be written as the countable intersection $\bigcap \left\lbrace \ioc{a-\frac{1}{n}}{b}\right\rbrace$, that every $\ioc{a}{b}$ can be written as the countable union $\bigcup \left\lbrace \icc{a+\frac{1}{n}}{b}\right\rbrace$, and to remember that a $\sigma$-algebra contains all the countable unions and countable intersections.

  Clearly $\sigma(\gE_1) = \sigma(\gE_4)$ and $\sigma(\gE_2) = \sigma(\gE_3)$, because every element of each first set is the complementary of an element of each second set, and vice versa.
  The inclusion $\sigma(\gE_1) \subseteq \sigma(\gA_1)$ follows by the fact that $\ioo{a}{+\infty}$ can be written as $\ioo{a}{a + 2} \cup \bigcup \ioo{a+n}{a+n+2}$. 
  Also the other inclusion is true because $\ioo{a}{b} = \ioo{a}{+\infty} \cap \bigcup \widetilde{\ioo{b-\frac{1}{n}}{+\infty}}$.

It remains to prove that $\sigma(\gE_1) = \sigma(\gE_2)$. 
This can be done observing that $\ioo{a}{+\infty} = \bigcup \ico{a+\frac{1}{n}}{+\infty}$ and that $\ico{a}{+\infty} = \bigcap \ioo{a-\frac{1}{n}}{+\infty}$.

The exercise is completed.
\end{proof}

\begin{exercise}[2016-10-10-1]
  Let us show that the Borel set $\Bor{\overline{\mathbb{R}}}$ coincides with the sets
  \begin{equation*}
    \mathfrak{S} \coloneqq \left\lbrace B \cup A \colon B \in \Bor{\mathbb{R}},\ A \subseteq \left\lbrace -\infty, +\infty\right\rbrace\right\rbrace
  \end{equation*}
  and also with the set
  \begin{equation*}
    \sigma(\mathfrak{B}) \coloneqq \sigma\left(\left\lbrace[a,+\infty] : a \in \mathbb{R}\right\rbrace \cup \left\lbrace[-\infty,+\infty]\right\rbrace\right).
  \end{equation*}
\end{exercise}
\begin{proof}
  First we are going to prove that $\mathfrak{S} = \sigma(\mathfrak{B})$.
  The inclusion $(\supseteq)$ easily follows from the fact that every set in $\mathfrak{B}$ belongs, by definition, also to $\mathfrak{S}$.
  In order to prove the other inclusion we will first prove that $\mathcal{B} \subseteq \sigma(\mathfrak{B})$, and the we will prove the same for $\left\lbrace -\infty, +\infty\right\rbrace$.
  From the previous exercise, if we prove that $\mathfrak{E_2} \subseteq \sigma(\mathfrak{B})$, since $\mathcal{B} = \sigma(\mathfrak{E_2})$, we obtain that $\mathcal{B} \subseteq \sigma(\mathfrak{B})$.
  It is sufficient to prove that $\left\lbrace +\infty\right\rbrace \in \mathfrak{B}$ since we know that
  \begin{equation*}
    \left\lbrace \ico{a}{+\infty} \cup \left\lbrace +\infty\right\rbrace \colon a \in \mathbb{R}\right\rbrace \subseteq \mathfrak{B}.
  \end{equation*}
  For this purpose let us write
  \begin{equation*}
    \left\lbrace+\infty\right\rbrace = \bigcap_{n \in \mathbb{N}}[n,+\infty].
  \end{equation*}

  It remains to prove that $\left\lbrace -\infty\right\rbrace \in \mathfrak{B}$.
  This is proved by the following:
  \begin{equation*}
    \left\lbrace-\infty\right\rbrace = [-\infty,+\infty] \setminus \bigcup_{n \in \mathbb{N}}\icc{a}{+\infty},
  \end{equation*}
  which is the difference of an element of $\mathfrak{B}$ and a countable union of elements of $\mathfrak{B}$.
 
  Let us now prove that $\Bor{\overline{\mathbb{R}}}$ coincides with the other two sets.
  First we will prove that $\Bor{\overline{\mathbb{R}}} \subseteq \mathfrak{S}$.
  In order to do this we remind who the open sets of $\overline{\mathbb{R}}$ are: if $A$ is an open set of $\overline{\mathbb{R}}$ then $A \cap \mathbb{R}$ is an open set of $\mathbb{R}$ and if $-\infty$ (respectively $+\infty$) is contained in $A$, then exists $a \in \mathbb{R}$ such that $[-\infty,a[ \subseteq A$ (respectively $]a,+\infty] \subseteq A$).
Since all the open sets of $\overline{\mathbb{R}}$ are clearly contained in $\mathfrak{S}$, the inclusion we want to prove follows.

Let us now prove that $\sigma(\mathfrak{B}) \subseteq \Bor{\overline{\mathbb{R}}}$.
This is obvious since the sets of $\mathfrak{B}$ are all open sets of $\overline{\mathbb{R}}$ and so the sigma-ring they generate is contained in $\Bor{\overline{\mathbb{R}}}$.

  This completes the proof.
\end{proof}

\begin{exercise}[insert date]  
  Sia $\emptyset \neq \gE_0 \subseteq \powerset(X)$ e per $\alpha \in \Ord$ sia 
\begin{equation}
  \gE_{\alpha}\coloneqq \left\{ \bigcup_{n\in \N} \left(A_n\setminus B_n \st A_n,B_n \in \bigcup_{\beta<\alpha}\gE_{\beta}\right)\right\}.
\end{equation}
Dimostrare allora che $\sigma(\gE_0)=\bigcup_{\alpha<\omega_1}\gE_{\alpha}$.
\end{exercise}
\begin{proof}
  Dimostriamo le due inclusioni.
  \begin{itemize}
    \item[$\supseteq$] Si può dimostrare utilizzando il principio di induzione transfinita, nella forma
    $$P(0)\land \forall \alpha>0 \left(\forall \beta<\alpha P(\beta) \implies P(\alpha) \right) \implies \forall \alpha P(\alpha)$$
    sulla proposizione $\sigma(\gE_0)\supseteq\gE_{\alpha}$.
    \begin{itemize}
      \item[$\alpha=0$] Vero per definizione di ($\sigma$-)anello generato.
      \item[$\alpha>0$] Per ipotesi induttiva $\sigma(\gE_0)\supseteq\gE_{\beta}\, \forall \beta<\alpha$, quindi $\sigma(\gE_0)\supseteq\bigcup_{\beta<\alpha}\gE_{\beta}$. Ma $\sigma(\gE_0)$ è chiuso per differenza e unione numerabile, quindi date due successioni $A_n, B_n \in \bigcup_{\beta<\alpha}\gE_{\beta}$, abbiamo che $\forall n \in \N\, C_n=A_n\setminus B_n\in \sigma(\gE_0)$, e $\bigcup_{n\in \N}C_n \in \sigma(\gE_0)$. Per ipotesi un generico elemento di $\gE_{\alpha}$ si scrive in questa forma, e dunque $\sigma(\gE_0)\supseteq\gE_{\alpha}$.
    \end{itemize}
    Allora $\sigma(\gE_0)$ contiene ogni $\gE_{\alpha}$, e quindi anche la loro unione.
    \item[$\subseteq$] Basta dimostrare che $S\coloneqq\bigcup_{\alpha<\omega_1}\gE_{\alpha}$ è un anello.
    \begin{itemize}
      \item[Differenza] Siano $A,B\in S$; allora $\exists \alpha_1, \alpha_2<\omega_1 \st A\in \gE_{\alpha_1}, B\in \gE_{\alpha_2}$, e sia $\alpha=\min\{\alpha_1, \alpha_2\}$. allora per ipotesi $A\setminus B\in\gE_{\alpha+1}$, con $\alpha+1<\omega_1$, perché $\omega_1$ non è un successore.      
      \item[Unione] Sia $(A_n)_{n\in\N}$ una successione si elementi di $S$, $B_n=\emptyset \, \forall n \in \N$, e sia $\alpha_n$ il più piccolo ordinale per cui $A_n\in \gE_{\alpha_n}$. Dato che $\cof(\omega_1)>\aleph_0\ \alpha=\bigcup_{n \in \N}\alpha_n<\omega_1$, e dunque per ipotesi $\bigcup_{n \in \N} A_n\in \gE_{\alpha}\subseteq S$
    \end{itemize}
  \end{itemize}
  Abbiamo anche dimostrato che da $\lambda=\omega_1$ in poi la successione $\bigcup_{\alpha<\lambda}\gE_{\alpha}$ è stazionaria.
\end{proof}
\end{document}

%%% Local Variables:
%%% mode: latex
%%% TeX-master: "../EserciziIstituzioniAnalisi"
%%% End:
