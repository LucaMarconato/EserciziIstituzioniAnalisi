\documentclass[../EserciziIstituzioniAnalisi.tex]{subfiles} 
\begin{document}

\begin{exercise}
  
\end{exercise}

\begin{exercise}[13/10/2016]
Sia $\gN$ un ideale di un anello $\gA$.
\begin{enumerate}
\item Descrivere l'insieme $\alg(\gN)$.
\item Dimostrare che se $\gN$ \`e un $\sigma$-ideale in $\gA$ e se $\gM$ \`e un $\sigma$-anello allora $\alg(\gM)$ \`e in realt\`a una $\sigma$-algebra.
\end{enumerate}
\end{exercise}
\begin{proof}
  Assegnato a Ilaria.
\end{proof}

\begin{exercise}[2016-10-13]
Sia $\emptyset \neq \gE \subseteq \powerset(X)$ e sia $\emptyset \neq A \subseteq X$.
Allora $\sigma(\gE \cap A) = \sigma(\gE) \cap A$ avendo definito $\gE \cap A \coloneqq \left\lbrace E \cap A \colon E \in \gE\right\rbrace$.
\begin{enumerate}
\item Descrivere l'insieme $\alg(\gN)$.
\item Dimostrare che se $\gN$ \`e un $\sigma$-ideale in $\gA$ e se $\gM$ \`e un $\sigma$-anello allora $\alg(\gM)$ \`e in realt\`a una $\sigma$-algebra.
\end{enumerate}
\end{exercise}
\begin{proof}
Assegnato a Gaetano.
\end{proof}

\begin{exercise}[13/10/2016]
  Sia $\emptyset \in \gN \subseteq \gA$ algebra in $\powerset(X)$ con $X \not\in \gN$.
  \begin{enumerate}
  \item 
  \end{enumerate}
  Definiamo la funzione $\nu(\cot)$ ponendo 
  \begin{equation*}
    \nu(A) \coloneqq
    \begin{cases}
      0 & \text{se $A \in \gN$}\\
      +\infty & \text{se $A \in \gA\setminus \gN$}.
    \end{cases}
  \end{equation*}
  Mostrare che $\nu$ \`e finitamente additiva se e solo se $\gN$ \`e un ideale. 
\item Definiamo ora
  \begin{equation*}
    \nu(A) \coloneqq
    \begin{cases}
      0 & \text{se $A \in \gN$}\\
      1 & \text{se $A \in \gA\setminus \gN$}.
    \end{cases}
  \end{equation*}
  Mostrare che $\nu$ è finitamente additiva se e solo se $\gN$ è un ideale massimale in $\gA$.
\end{exercise}
\begin{proof}
La prima parte è stata svolta da Stefano, la seconda è da svolgere.
\end{proof}
\end{document}

%%% Local Variables:
%%% mode: latex
%%% TeX-master: "../EserciziIstituzioniAnalisi"
%%% End:
