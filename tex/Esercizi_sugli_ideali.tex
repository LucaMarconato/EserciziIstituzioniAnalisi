\documentclass[../EserciziIstituzioniAnalisi.tex]{subfiles} 
\begin{document}
\begin{exercise}[13/10/2016]
Sia $\gN$ un ideale di un anello $\gA$.
\begin{enumerate}
\item Descrivere l'insieme $\alg(\gN)$.
\item Dimostrare che se $\gN$ è un $\sigma$-ideale in $\gA$ e se $\gM$ è un $\sigma$-anello allora $\alg(\gM)$ è in realtà una $\sigma$-algebra.
\end{enumerate}
\end{exercise}
\begin{proof}
  Assegnato a Ilaria.
\end{proof}

\begin{exercise}[2016-10-13]
Sia $\emptyset \neq \gE \subseteq \powerset(X)$ e sia $\emptyset \neq A \subseteq X$.
Allora $\sigma(\gE \cap A) = \sigma(\gE) \cap A$ avendo definito $\gE \cap A \coloneqq \left\lbrace E \cap A \colon E \in \gE\right\rbrace$.
\begin{enumerate}
\item Descrivere l'insieme $\alg(\gN)$.
\item Dimostrare che se $\gN$ è un $\sigma$-ideale in $\gA$ e se $\gM$ è un $\sigma$-anello allora $\alg(\gM)$ è in realtà una $\sigma$-algebra.
\end{enumerate}
\end{exercise}
\begin{proof}
Assegnato a Gaetano.
\end{proof}

\begin{exercise}[13/10/2016]
  Sia $\emptyset \in \gN \subseteq \gA$ algebra in $\powerset(X)$ con $X \not\in \gN$.
  \begin{enumerate}
  \item 
  \end{enumerate}
  Definiamo la funzione $\nu(\cot)$ ponendo 
  \begin{equation*}
    \nu(A) \coloneqq
    \begin{cases}
      0 & \text{se $A \in \gN$}\\
      +\infty & \text{se $A \in \gA\setminus \gN$}.
    \end{cases}
  \end{equation*}
  Mostrare che $\nu$ è finitamente additiva se e solo se $\gN$ è un ideale. 
\item Definiamo ora
  \begin{equation*}
    \nu(A) \coloneqq
    \begin{cases}
      0 & \text{se $A \in \gN$}\\
      1 & \text{se $A \in \gA\setminus \gN$}.
    \end{cases}
  \end{equation*}
  Mostrare che $\nu$ è finitamente additiva se e solo se $\gN$ è un ideale massimale in $\gA$.
\end{exercise}
\begin{proof}
  La prima parte è stata svolta da Stefano, la seconda è da svolgere.


  Seconda parte:
  Sappiamo già che gli insiemi di misura nulla formano un ideale. Resta quindi da dimostrare che $\nu$ è finitamente additiva se e solo se questo ideale è massimale.
  \begin{itemize}
    \item[$\Rightarrow$] Supponiamo che la misura sia finitamente additiva e per assurdo $\gN$ non sia massimale. Allora $\exists A\in\gA \st A\notin \gN, \compl{A}\notin\gN$. Allora $\nu(X)=\nu(A\sqcup \compl{A})=\nu(A)+\nu (\compl{A})=2$, assurdo.
    \
    \item[$\Leftarrow$] Sia ora $\gN$ massimale, e consideriamo $A,B\in\gA$ disgiunti. Ci sono tre possibilità:
    \begin{itemize}
      \item Se $A,B\in\gN$, $\nu(A)=\nu(B)=\nu(A \cup B)=0$ e la finita additività vale.
      \item Se esattamente uno tra $A$ e $B$ è un elemento di $\gN$, $A\cup B \notin \gN$, perché l'ideale è chiuso per sottoinsiemi. In questo caso $\nu(A)+\nu(B)=1+0=\nu(A \cup B)$ e la finita additività vale.
      \item Se $A,B\notin\gN,\ \nu(A)+\nu(B)=2>\nu(A \cup B)$. Mostriamo allora che questo caso non può verificarsi. Per la massimalità dell'ideale $\compl{A},\compl{B}\in\gN$ e  inoltre $\gN \ni \compl{A} \cup \compl{B}=\compl{A\cap B}=\compl{\emptyset}=X$, assurdo perché per definizione gli ideali massimali sono propri.
    \end{itemize}
  \end{itemize}
\end{proof}
\end{document}

%%% Local Variables:
%%% mode: latex
%%% TeX-master: "../EserciziIstituzioniAnalisi"
%%% End:
