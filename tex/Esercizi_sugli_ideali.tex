\documentclass[../EserciziIstituzioniAnalisi.tex]{subfiles} 
\begin{document} %label prefix: ideal:
\begin{exercise}[2016-10-13]
  Sia $\emptyset \neq \gN \subseteq \gA$ anello.
  Mostrare che le seguenti affermazioni sono equivalenti:
  \begin{enumerate}
    \item \label{ideal:algebraIdealDefinition} $\gN$ \`e un ideale in $\gA$ (nel senso di Algebra I)
    \item \label{ideal:setTheoryIdealDefinition} $\gN$ è un ideale secondo la definizionedi teoria degli insiemi, ossia se valgono le condizioni seguenti:
    \begin{subnumcases}{}
      A \subseteq B \in \gN,\ A \in \gA \Rightarrow A \in \gN \label{ideal:setTheoryIdealDefinitionC1}
      \\
      A, B \in \gN \Rightarrow A \cup B \in \gN \label{ideal:setTheoryIdealDefinitionC2}
    \end{subnumcases}
    \item \label{ideal:varIdealDefinition} se $A \in \gA,\ B, C \in \gN$ e $A\subseteq B \cup C$ allora $A \in \gN$.
  \end{enumerate}
\end{exercise}
\begin{proof}
  Mostriamo la catena di implicazioni $2\implies 3\implies 1\implies 2$.
  \begin{itemize}
    \item[$2\implies 3$] Siano $A,B,C$ come nelle ipotesi del \cref{ideal:varIdealDefinition}. Allora per ipotesi vale \ref{ideal:setTheoryIdealDefinitionC2}, e quindi $B\cup C\in \gN$.
    Usando quindi il fatto che $A\subseteq B\cup C$ e l'ipotesi \ref{ideal:setTheoryIdealDefinitionC1}, si ottiene $A\in \gN$.
    \item[$3\implies 1$] Siano $A\in\gA, B\in\gN$; allora $\gA\ni A\cap B\subseteq B\cup B$ dunque per il \cref{ideal:varIdealDefinition} $A\cap B\in\gN$.
    
    Siano ora $A,B\in\gN$; allora per definizione di differenza simmetrica $\gA\ni A\Delta B\subseteq A\cup B$ e dunque per il \cref{ideal:varIdealDefinition} $A\Delta B\in\gN$.
  \end{itemize}
\end{proof}

\begin{exercise}[13/10/2016]
  Sia $\gN$ un ideale di un anello $\gA$.
\begin{enumerate}
\item Descrivere l'insieme $\alg(\gN)$.
\item Dimostrare che se $\gN$ è un $\sigma$-ideale in $\gA$ e se $\gM$ è un $\sigma$-anello allora $\alg(\gM)$ è in realtà una $\sigma$-algebra.
\end{enumerate}
\end{exercise}
\begin{proof}
  Assegnato a Ilaria.
\end{proof}

\begin{exercise}[2016-10-13]
Sia $\emptyset \neq \gE \subseteq \powerset(X)$ e sia $\emptyset \neq A \subseteq X$.
Allora $\sigma(\gE \cap A) = \sigma(\gE) \cap A$ avendo definito $\gE \cap A \coloneqq \left\lbrace E \cap A \colon E \in \gE\right\rbrace$.
\begin{enumerate}
\item Descrivere l'insieme $\alg(\gN)$.
\item Dimostrare che se $\gN$ è un $\sigma$-ideale in $\gA$ e se $\gM$ è un $\sigma$-anello allora $\alg(\gM)$ è in realtà una $\sigma$-algebra.
\end{enumerate}
\end{exercise}
\begin{proof}
Assegnato a Gaetano.
\end{proof}

\begin{exercise}[13/10/2016]
  Le seguenti due proprietà sono equivalenti per un ideale $\gN$ in un'algebra di insiemi $\gA \subseteq \PS(X)$.
  \begin{enumerate}
    \item \label{ideal:algebra} Essere massimale nella famiglia degli ideali propri. 
    \item \label{ideal:analysis} Essere un ideale massimale come nella definizione per famiglie di insiemi, cioè $\forall A\in \gA,\ A\in\gN \lor \compl{A}\in\gN$.
  \end{enumerate}
\end{exercise}
\begin{proof}
  \begin{itemize}
    \item[$\Rightarrow$] Sappiamo per un teorema di algebra che $\frac{\gA}{\gN}$ è un campo. L'elemento unitario di $\gA$ è $X$, e quindi la classe $X+\gN$ è l'unità del campo.  
    Questo significa che
    \begin{equation*}
      \forall A \in \gA, A \notin \gN\ \exists B \st A+\gN \cdot B+\gN = X+\gN
    \end{equation*}
    cioè
    \begin{equation*}
      A \cap B \in X \Delta \gN \coloneqq \{D\in \gA \st \exists N\in \gN, D = X\setminus N \}
    \end{equation*}
    e quindi $A$ coincide con $X$ a meno di un elemento dell'ideale $\gN$, ossia $\compl{A}\in \gN$.
    
    \item[$\Leftarrow$] Supponiamo per assurdo che non valga il \cref{ideal:algebra}: allora esiste $A\notin \gN$ tale che l'ideale $\gN'$ generato da $\gN\cup \{A\}$ è proprio. Per il \cref{ideal:analysis} $\compl{A}\in \gN$ e quindi, dato che un ideale è chiuso per unione $X=A\cup\compl{A}\in \gN'$, e poiché gli ideali sono chiusi per contenimento $\gN'=\gA$, assurdo.
  \end{itemize}
\end{proof}
\begin{remark}
  Il campo quoziente dell'esercizio precedente è $\mathbb{F}_2$, infatti ogni elemento non in $\gN$ è nella classe di $X$.
\end{remark}

\begin{exercise}[13/10/2016]
  Sia $\emptyset \in \gN \subseteq \gA$ algebra in $\powerset(X)$ con $X \not\in \gN$. Definiamo la funzione $\nu(\cdot)$ ponendo
  \begin{enumerate}
    \item 
    \begin{equation*}
      \nu(A) \coloneqq
      \begin{cases}
        0 & \text{se $A \in \gN$}\\
        +\infty & \text{se $A \in \gA\setminus \gN$}.
      \end{cases}
    \end{equation*}
    Mostrare che $\nu$ è finitamente additiva se e solo se $\gN$ è un ideale. 
    \item Definiamo ora
    \begin{equation*}
      \nu(A) \coloneqq
      \begin{cases}
        0 & \text{se $A \in \gN$}\\
        1 & \text{se $A \in \gA\setminus \gN$}.
      \end{cases}
    \end{equation*}
    Mostrare che $\nu$ è finitamente additiva se e solo se $\gN$ è un ideale massimale in $\gA$.
  \end{enumerate}
\end{exercise}
\begin{proof}
  \begin{enumerate}
    \item La prima parte è stata svolta da Stefano.
    \item Sappiamo già che gli insiemi di misura nulla formano un ideale. Resta quindi da dimostrare che $\nu$ è finitamente additiva se e solo se questo ideale è massimale.
    \begin{itemize}
      \item[$\Rightarrow$] Supponiamo che la misura sia finitamente additiva e per assurdo $\gN$ non sia massimale. Allora $\exists A\in\gA \st A\notin \gN, \compl{A}\notin\gN$. Allora $\nu(X)=\nu(A\sqcup \compl{A})=\nu(A)+\nu (\compl{A})=2$, assurdo.
      \item[$\Leftarrow$] Sia ora $\gN$ massimale, e consideriamo $A,B\in\gA$ disgiunti. Ci sono tre possibilità:
      \begin{itemize}
        \item Se $A,B\in\gN$, $\nu(A)=\nu(B)=\nu(A \cup B)=0$ e la finita additività vale.
        \item Se esattamente uno tra $A$ e $B$ è un elemento di $\gN$, $A\cup B \notin \gN$, perché l'ideale è chiuso per sottoinsiemi. In questo caso $\nu(A)+\nu(B)=1+0=\nu(A \cup B)$ e la finita additività vale.
        \item Se $A,B\notin\gN,\ \nu(A)+\nu(B)=2>\nu(A \cup B)$. Dimostriamo allora per assurdo che questo caso non può verificarsi. Per la massimalità dell'ideale $\compl{A},\compl{B}\in\gN$ e  inoltre $\gN \ni \compl{A} \cup \compl{B}=\compl{A\cap B}=\compl{\emptyset}=X$, assurdo perché per definizione gli ideali massimali sono propri.
      \end{itemize}
    \end{itemize}
  \end{enumerate}
\end{proof}
\end{document}

%%% Local Variables:
%%% mode: latex
%%% TeX-master: "../EserciziIstituzioniAnalisi"
%%% End:
